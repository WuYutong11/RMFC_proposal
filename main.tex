%
%                       This is a basic LaTeX Template
%                       for the Informatics Research Review

\documentclass[a4paper,11pt]{article}
% Add local fullpage and head macros
\usepackage{head,fullpage}     
% Add graphicx package with pdf flag (must use pdflatex)
\usepackage[pdftex]{graphicx}  
% Better support for URLs
\usepackage{url}
% Date formating
\usepackage{datetime}
% For Gantt chart
\usepackage{pgfgantt}
\usepackage{xcolor}
\usepackage[utf8]{inputenc}

\newdateformat{monthyeardate}{%
  \monthname[\THEMONTH] \THEYEAR}

\parindent=0pt          %  Switch off indent of paragraphs 
\parskip=5pt            %  Put 5pt between each paragraph  
\Urlmuskip=0mu plus 1mu %  Better line breaks for URLs


%                       This section generates a title page
%                       Edit only the following three lines
%                       providing your exam number, 
%                       the general field of study you are considering
%                       for your review, and name of IRR tutor

\newcommand{\examnumber}{B264369}
\newcommand{\field}{Intelligent Regulatory Advisory Platform for FCA Consumer Duty Compliance Powered by DeepSeek-V3 and Llama 3}
\newcommand{\tutor}{Waylon Li}
\newcommand{\supervisor}{Fengxiang He}

\begin{document}
\begin{minipage}[b]{110mm}
        {\Huge\bf School of Informatics
        \vspace*{17mm}}
\end{minipage}
\hfill
\begin{minipage}[t]{40mm}               
        \makebox[40mm]{
        \includegraphics[width=40mm]{crest.png}}
\end{minipage}
\par\noindent
    % Centre Title, and name
\vspace*{2cm}
\begin{center}
        \Large\bf Informatics Project Proposal \\
        \Large\bf \field
\end{center}
\vspace*{1.5cm}
\begin{center}
        \bf \examnumber\\
        \monthyeardate\today
\end{center}
\vspace*{5mm}

%
%                       Insert your abstract HERE
%                       
\begin{abstract}
        The abstract is a short concise outline of your 
        project proposal, {\bf of no more than around 100 words}.
\end{abstract}

\vspace*{1cm}

\vspace*{3cm}
Date: \today

\vfill
{\bf Tutor:} \tutor\\
{\bf Supervisor:} \supervisor
\newpage

%                                               Through page and setup 
%                                               fancy headings
%\setcounter{page}{1}                            % Set page number to 1
%\footruleheight{1pt}
%\headruleheight{1pt}
%\lfoot{\small School of Informatics}
%\lhead{Informatics Research Review}
%\rhead{- \thepage}
%\cfoot{}
%\rfoot{Date: \date{\today}}
%


\section{Motivation}

With the rapid development of FinTech, the product structure, interaction mode and data management mode of financial services are constantly evolving, and the traditional financial regulatory model faces severe challenges in adaptability and response efficiency. In order to improve the level of financial consumer protection, the Financial Conduct Authority (FCA) of the United Kingdom officially introduced the Consumer Duty regulatory framework in 2022, requiring financial institutions to take "customer-centricity" as the core principle throughout the life cycle from product design, sales to customer service to ensure that customers get good results (Financial Conduct Authority, 2022).

However, due to the high complexity and professionalism of the FCA handbook, the relevant clauses often involve abstract legal language, cross-references and compliance judgment standards, which makes it difficult for enterprises to understand and implement the regulatory framework. At present, most financial institutions still rely mainly on manual reading and expert interpretation in compliance review, which is not only inefficient, but also difficult to ensure consistency and comprehensiveness (Breden, Studer and Woelke, 2020). Therefore, building digital auxiliary tools for regulatory content has become an urgent research issue.

In the field of artificial intelligence, the large language models (LLMs) and their derived retrieval-
augmented generation (RAG) technologies that have developed rapidly in recent years have provided technical breakthroughs for regulatory understanding, text matching, and compliance recommendations. DeepSeek-V3 is the latest open source semantic retrieval model with powerful multi-round semantic matching capabilities, which can efficiently locate legal provisions in massive unstructured texts (DeepSeek AI, 2023); Llama 3, as a high-performance language model launched by Meta, has demonstrated powerful capabilities in regulatory question-answering, provision summarization, and generative reasoning tasks (Meta AI, 2024). The organic combination of these two models to build an intelligent consulting platform dedicated to FCA consumer obligations terms is not only of theoretical exploration significance, but also has practical application prospects.

\subsection{Problem Statement}

At present, financial institutions have obvious problems in implementing the FCA's consumer duty compliance requirements. First, the structure of the regulations is huge and the content is frequently updated. It is difficult for institutions to grasp the latest regulations in a timely manner. Compliance interpretation relies heavily on the subjective judgment of experts, resulting in differences in understanding and inconsistent operations. Moreover, traditional document retrieval tools are often based on keyword matching, which makes it difficult to support semantic problem location and article connection, and it is difficult to meet the FCA's new requirements for "active compliance" and "customer-centricity" (Financial Conduct Authority, 2023).

Therefore, it is urgent to develop an intelligent compliance assistance platform that can automatically retrieve and understand the relevant provisions of the FCA consumer duty, and provide compliance personnel with popular, accurate and targeted compliance guidance through semantic matching and generative technology to solve the "last mile" problem of understanding financial services compliance.

\subsection{Research Hypothesis and Objectives}

The core hypothesis of research is that by combining the efficient semantic retrieval capabilities of DeepSeek-V3 with the natural language understanding and generation capabilities of Llama 3, an intelligent consulting platform that integrates regulatory query and intelligent question-answering can be built, significantly improving the responsiveness and compliance efficiency of financial institutions to the FCA consumer duty clauses.

The research objectives include:
(1) Build a semantic retrieval module for FCA consumer duty based on DeepSeek-V3 to quickly locate targeted provisions;
(2) Develop a natural language question-answering system based on Llama 3;
(3) Integrate the platform prototype and complete test verification;
(4) Evaluate the effectiveness of the platform in real compliance scenarios.

The research will be strictly limited to the content directly related to consumer duty in the FCA handbook, and will not involve a wide range of financial products, anti-money laundering or capital regulatory provisions to ensure the pertinence and depth of the research.

\subsection{Timeliness and Novelty}

The FCA's Consumer Duty will come into full effect in July 2023 and has become one of the core issues of financial compliance in the UK. Most small and medium-sized financial institutions lack sufficient legal teams and compliance resources, and face tremendous pressure to cope with such new regulations. Therefore, it is of great practical value and urgency to develop a compliance intelligent platform that can be deployed immediately, is cost-controlled, and supports user interaction.

In this background, semantically enhanced regulatory retrieval systems (such as DeepSeek-V3) and large language model-driven generative question-answering systems (such as Llama 3) provide unprecedented technical possibilities. This type of retrieval-augmented generation (RAG) architecture is considered to be an important direction for promoting high-trust AI decision-making systems (Lewis et al., 2020). Compared with traditional keyword matching or rule-based compliance systems, RAG has stronger text understanding and task generalization capabilities, and is suitable for retrieval and generation scenarios of regulatory unstructured text (Arner, Barberis and Buckley, 2017). Although RAG technology has been initially explored in the fields of law, education, and medicine, there is currently no literature that systematically applies it to the specific clause analysis tasks of UK consumer financial regulation, especially the compliance intelligent platform for FCA consumer obligations clauses is still in a blank stage. Therefore, this study combines semantic retrieval and generation models in terms of methods, focuses on policy hotspots in terms of application, and has clear timeliness and technological innovation value.

\subsection{Significance}

This study expands the research path in the intersection of artificial intelligence and financial regulation, demonstrates the adaptability of LLMs in legal terms and question-answer generation, and enriches the technical method system of RegTech. In practice, this study is expected to provide financial institutions with a set of feasible compliance automation tools, especially for small and medium-sized banks, insurance companies, wealth management platforms, etc., which are under resource-constrained regulatory response. At the social level, the platform helps to improve the transparency and consistency of regulatory enforcement, enhance the trust of financial consumers, and thus promote the stability and healthy development of the entire financial system.

\subsection{Feasibility}

This study is based on existing open source tools. DeepSeek-V3 and Llama 3 have been released and have good scalability and interface documentation, which are suitable for rapid experiments in local deployment or cloud environments. In addition, the FCA's regulatory text is open and transparent, with a structured format and good data accessibility. The project is planned to last for 3 months, including data preparation, platform construction, test optimization and paper writing, and the time and technology are controllable.

\subsection{Beneficiaries}

The results of this study will directly benefit the compliance departments and legal teams of financial institutions, helping to improve compliance enforcement and the efficiency of understanding provisions. Regulators can use similar systems to strengthen the uniformity and transparency of regulatory interpretation. At the same time, consumers will benefit from a clearer and more reliable financial product service experience. At the academic level, this research can provide a replicable paradigm and data samples for the future development of a generative AI-based regulatory interpretation system, supporting subsequent horizontal expansion to other regulatory areas such as anti-money laundering, open banking, and ESG.

\section{Background and Related Work}

Demonstrate a knowledge and understanding of past and current work in the subject area, including relevant references like this \cite{template}.


\section{Programme and Methodology}

\begin{itemize}
    \item Detail the methodology to be used in pursuit of the research and justify this choice.
    \item Describe your contributions and novelty and where you
    will go beyond the state-of-the-art (new methods, new tools,
    new data, new insights, new proofs,...)
    \item Describe the programme of work, indicating the research to be undertaken and the milestones that can be used to measure its progress.
    \item Where suitable define work packages and define the dependences
    between these work packages. WPs and their dependences should be
    shown in the Gantt chart in the research plan.
    \item Explain how the project will be managed.
    \item State the limitations of your research.
\end{itemize}

\subsection{Risk Assessment}

\subsection{Ethics}


\section{Evaluation}

\begin{itemize}
    \item Describe the specific methods of data collection.
    \item Explain how you intent to analyse and interpret the results.
\end{itemize}

\section{Expected Outcomes}

Conclude your research proposal by addressing your predicted outcomes. What are you hoping to prove/disprove? Indicate how you envisage your research will contribute to debates and discussions in your particular subject area:

\begin{itemize}
    \item How will your research make an original contribution to knowledge?
    \item How might it fill gaps in existing work? 
    \item How might it extend understanding of particular topics?
\end{itemize}


\section{Research Plan, Milestones and Deliverables}

\definecolor{barblue}{RGB}{153,204,254}
\definecolor{groupblue}{RGB}{51,102,254}
\definecolor{linkred}{RGB}{165,0,33} 

\begin{figure}[htbp]
\begin{ganttchart}[
    y unit title=0.4cm,
    y unit chart=0.5cm,
    vgrid,hgrid,
    x unit=1.55mm,
    time slot format=isodate,
    title/.append style={draw=none, fill=barblue},
    title label font=\sffamily\bfseries\color{white},
    title label node/.append style={below=-1.6ex},
    title left shift=.05,
    title right shift=-.05,
    title height=1,
    bar/.append style={draw=none, fill=groupblue},
    bar height=.6,
    bar label font=\normalsize\color{black!50},
    group right shift=0,
    group top shift=.6,
    group height=.3,
    group peaks height=.2,
    bar incomplete/.append style={fill=green}
   ]{2018-06-01}{2018-08-16}
   \gantttitlecalendar{month=name}\\
   \ganttbar[
    progress=100,
    bar progress label font=\small\color{barblue},
    bar progress label node/.append style={right=4pt},
    bar label font=\normalsize\color{barblue},
    name=pp
   ]{Background Reading}{2018-06-01}{2018-06-14} \\
\ganttset{progress label text={}, link/.style={black, -to}}
\ganttgroup{Work Package 1}{2018-06-14}{2018-06-30} \\
\ganttbar[progress=4, name=T1A]{Task A}{2018-06-14}{2018-06-21} \\
\ganttlinkedbar[progress=0]{Task B}{2018-06-21}{2018-06-30} \\
\ganttgroup{Work Package 2}{2018-07-01}{2018-07-14} \\
\ganttbar[progress=15, name=T2A]{Task A}{2018-07-01}{2018-07-07} \\
\ganttlinkedbar[progress=0]{Task B}{2018-07-07}{2018-07-14} \\
\ganttgroup{Dissertation}{2018-07-14}{2018-08-16} \\
  \ganttbar[progress=0]{Task A}{2018-07-14}{2018-08-16}
  \ganttset{link/.style={green}}
  \ganttlink[link mid=.4]{pp}{T1A}
  \ganttlink[link mid=.159]{pp}{T2A}
\end{ganttchart}
\caption{Gantt Chart of the activities defined for this project.}
\label{fig:gantt}
\end{figure}

\begin{table}[htbp]
    \begin{center}
        \begin{tabular}{|c|c|l|}
        \hline
        \textbf{Milestone} & \textbf{Week} & \textbf{Description} \\
        \hline
        $M_1$ & 2 & Feasibility study completed \\
        $M_2$ & 5 & First prototype implementation completed \\
        $M_3$ & 7 & Evaluation completed \\
        $M_4$ & 10 & Submission of dissertation \\
        \hline
        \end{tabular} 
    \end{center}
    \caption{Milestones defined in this project.}
    \label{fig:milestones}
\end{table}

\begin{table}[htbp]
    \begin{center}
        \begin{tabular}{|c|c|l|}
        \hline
        \textbf{Deliverable} & \textbf{Week} & \textbf{Description} \\
        \hline
        $D_1$ & 6 & Software tool for \dots\\
        $D_2$ & 8 & Evaluation report on \dots\\
        $D_3$ & 10 & Dissertation \\
        \hline
        \end{tabular} 
    \end{center}
    \caption{List of deliverables defined in this project.}
    \label{fig:deliverables}
\end{table}


%                Now build the reference list
\bibliographystyle{unsrt}   % The reference style
%                This is plain and unsorted, so in the order
%                they appear in the document.

{\small
\bibliography{main}       % bib file(s).
}
\end{document}

