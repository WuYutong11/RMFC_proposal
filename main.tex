%
%                       This is a basic LaTeX Template
%                       for the Informatics Research Review

\documentclass[a4paper,11pt]{article}
% Add local fullpage and head macros
\usepackage{head,fullpage}      
% Add graphicx package with pdf flag (must use pdflatex)
\usepackage[pdftex]{graphicx}  
% Better support for URLs
\usepackage{url}
% Date formating
\usepackage{datetime}
% For Gantt chart
\usepackage{pgfgantt}
\usepackage{xcolor}
\usepackage[utf8]{inputenc}

\newdateformat{monthyeardate}{%
  \monthname[\THEMONTH] \THEYEAR}

\parindent=0pt          %  Switch off indent of paragraphs 
\parskip=5pt            %  Put 5pt between each paragraph  
\Urlmuskip=0mu plus 1mu %  Better line breaks for URLs


%                       This section generates a title page
%                       Edit only the following three lines
%                       providing your exam number, 
%                       the general field of study you are considering
%                       for your review, and name of IRR tutor

\newcommand{\examnumber}{B264369}
\newcommand{\field}{Intelligent Regulatory Advisory Platform for FCA Consumer Duty Compliance Powered by DeepSeek-V3 and Llama 3}
\newcommand{\tutor}{Waylon Li}
\newcommand{\supervisor}{Fengxiang He}

\begin{document}
\begin{minipage}[b]{110mm}
        {\Huge\bf School of Informatics
        \vspace*{17mm}}
\end{minipage}
\hfill
\begin{minipage}[t]{40mm}               
        \makebox[40mm]{
        \includegraphics[width=40mm]{crest.png}}
\end{minipage}
\par\noindent
    % Centre Title, and name
\vspace*{2cm}
\begin{center}
        \Large\bf Informatics Project Proposal \\
        \Large\bf \field
\end{center}
\vspace*{1.5cm}
\begin{center}
        \bf \examnumber\\
        \monthyeardate\today
\end{center}
\vspace*{5mm}

%
%                       Insert your abstract HERE
%                       
\begin{abstract}
        This research proposes an intelligent compliance consulting platform leveraging DeepSeek-V3 and Llama 3 to enhance regulatory understanding and implementation of the FCA’s Consumer Duty provisions. By integrating semantic retrieval and generative question-answering in a Retrieval-Augmented Generation (RAG) architecture, the platform aims to address the inefficiency and inconsistency of traditional compliance methods. The research demonstrates the feasibility and value of applying large language models to financial regulation, filling a significant research gap in RegTech and contributing to the digital transformation of consumer protection in UK financial services.
\end{abstract}

\vspace*{1cm}

\vspace*{3cm}
Date: \today

\vfill
{\bf Tutor:} \tutor\\
{\bf Supervisor:} \supervisor
\newpage

%                                               Through page and setup 
%                                               fancy headings
%\setcounter{page}{1}                            % Set page number to 1
%\footruleheight{1pt}
%\headruleheight{1pt}
%\lfoot{\small School of Informatics}
%\lhead{Informatics Research Review}
%\rhead{- \thepage}
%\cfoot{}
%\rfoot{Date: \date{\today}}
%


\section{Motivation}

With the rapid development of FinTech, the product structure, interaction mode and data management mode of financial services are constantly evolving, and the traditional financial regulatory model faces severe challenges in adaptability and response efficiency. In order to improve the level of financial consumer protection, the Financial Conduct Authority (FCA) of the United Kingdom officially introduced the Consumer Duty regulatory framework in 2022, requiring financial institutions to take "customer-centricity" as the core principle throughout the life cycle from product design, sales to customer service to ensure that customers get good results \cite{fca2022consumerduty}.

However, due to the high complexity and professionalism of the FCA handbook, the relevant clauses often involve abstract legal language, cross-references and compliance judgment standards, which makes it difficult for enterprises to understand and implement the regulatory framework. At present, most financial institutions still rely mainly on manual reading and expert interpretation in compliance review, which is not only inefficient, but also difficult to ensure consistency and comprehensiveness \cite{breden2020ai}. Therefore, building digital auxiliary tools for regulatory content has become an urgent research issue.

In the field of artificial intelligence, the large language models (LLMs) and their derived retrieval-augmented generation (RAG) technologies that have developed rapidly in recent years have provided technical breakthroughs for regulatory understanding, text matching, and compliance recommendations. DeepSeek-V3 is the latest open source semantic retrieval model with powerful multi-round semantic matching capabilities, which can efficiently locate legal provisions in massive unstructured texts \cite{deepseek2023}; Llama 3, as a high-performance language model launched by Meta, has demonstrated powerful capabilities in regulatory question-answering, provision summarization, and generative reasoning tasks \cite{meta2024llama3}. The organic combination of these two models to build an intelligent consulting platform dedicated to FCA consumer obligations terms is not only of theoretical exploration significance, but also has practical application prospects.


\subsection{Problem Statement}

At present, financial institutions have obvious problems in implementing the FCA's consumer duty compliance requirements. First, the structure of the regulations is huge and the content is frequently updated. It is difficult for institutions to grasp the latest regulations in a timely manner. Compliance interpretation relies heavily on the subjective judgment of experts, resulting in differences in understanding and inconsistent operations. Moreover, traditional document retrieval tools are often based on keyword matching, which makes it difficult to support semantic problem location and article connection, and it is difficult to meet the FCA's new requirements for "active compliance" and "customer-centricity" (Financial Conduct Authority, 2023).

Therefore, it is urgent to develop an intelligent compliance assistance platform that can automatically retrieve and understand the relevant provisions of the FCA consumer duty, and provide compliance personnel with popular, accurate and targeted compliance guidance through semantic matching and generative technology to solve the "last mile" problem of understanding financial services compliance.

\subsection{Research Hypothesis and Objectives}

The core hypothesis of research is that by combining the efficient semantic retrieval capabilities of DeepSeek-V3 with the natural language understanding and generation capabilities of Llama 3, an intelligent consulting platform that integrates regulatory query and intelligent question-answering can be built, significantly improving the responsiveness and compliance efficiency of financial institutions to the FCA consumer duty clauses.

The research objectives include:
(1) Build a semantic retrieval module for FCA consumer duty based on DeepSeek-V3 to quickly locate targeted provisions;
(2) Develop a natural language question-answering system based on Llama 3;
(3) Integrate the platform prototype and complete test verification;
(4) Evaluate the effectiveness of the platform in real compliance scenarios.

The research will be strictly limited to the content directly related to consumer duty in the FCA handbook, and will not involve a wide range of financial products, anti-money laundering or capital regulatory provisions to ensure the pertinence and depth of the research.

\subsection{Timeliness and Novelty}

The FCA's Consumer Duty will come into full effect in July 2023 and has become one of the core issues of financial compliance in the UK. Most small and medium-sized financial institutions lack sufficient legal teams, compliance resources and face tremendous pressure to cope with such new regulations. Therefore, it is of great practical value and urgency to develop a compliance intelligent platform that can be deployed immediately, is cost-controlled, and supports user interaction.

In this background, semantically enhanced regulatory retrieval systems (such as DeepSeek-V3) and large language model-driven generative question-answering systems (such as Llama 3) provide unprecedented technical possibilities. This type of retrieval-augmented generation (RAG) architecture is considered to be an important direction for promoting high-trust AI decision-making systems \cite{lewis2020rag}. Compared with traditional keyword matching or rule-based compliance systems, RAG has stronger text understanding and task generalization capabilities, and is suitable for retrieval and generation scenarios of regulatory unstructured text \cite{arner2017regtech}. Although RAG technology has been initially explored in the fields of law, education, and medicine, there is currently no literature that systematically applies it to the specific clause analysis tasks of UK consumer financial regulation, especially the compliance intelligent platform for FCA consumer obligations clauses is still in a blank stage. Therefore, this study combines semantic retrieval and generation models in terms of methods, focuses on policy hotspots in terms of application, and has clear timeliness and technological innovation value.

\subsection{Significance}

This study expands the research path in the intersection of artificial intelligence and financial regulation, demonstrates the adaptability of LLMs in legal terms and question-answer generation, and enriches the technical method system of RegTech. In practice, this study is expected to provide financial institutions with a set of feasible compliance automation tools, especially for small and medium-sized banks, insurance companies, wealth management platforms, etc., which are under resource-constrained regulatory response. At the social level, the platform helps to improve the transparency and consistency of regulatory enforcement, enhance the trust of financial consumers, and thus promote the stability and healthy development of the entire financial system.

\subsection{Feasibility}

This study is based on existing open source tools. DeepSeek-V3 and Llama 3 have been released and have good scalability and interface documentation, which are suitable for rapid experiments in local deployment or cloud environments. In addition, the FCA's regulatory text is open and transparent, with a structured format and good data accessibility. The project is planned to last for 3 months, including data preparation, platform construction, test optimization and paper writing, and the time and technology are controllable.

\subsection{Beneficiaries}

The results of this study will directly benefit the compliance departments and legal teams of financial institutions, helping to improve compliance enforcement and the efficiency of understanding provisions. Regulators can use similar systems to strengthen the uniformity and transparency of regulatory interpretation. At the same time, consumers will benefit from a clearer and more reliable financial product service experience. At the academic level, this research can provide a replicable paradigm and data samples for the future development of a generative AI-based regulatory interpretation system, supporting subsequent horizontal expansion to other regulatory areas such as anti-money laundering, open banking, and ESG.

\section{Background and Related Work}

With the breakthrough of artificial intelligence technology in the field of natural language processing, new development opportunities have been ushered in for tasks with high semantic complexity, such as legal text understanding and regulatory document parsing. In particular, large language models (LLMs) and semantic retrieval mechanisms based on deep learning have been widely used in tasks such as contract review \cite{zhong2020jec}, legal question answering \cite{chalkidis2020legalbert}, and financial rule parsing \cite{zhang2023regtech}.

In the legal field, researchers have developed a variety of specialized language models. For example, Legal-BERT has been fine-tuned for multilingual and case law corpora, and significantly outperforms general models in tasks such as legislative interpretation and legal article alignment \cite{chalkidis2020legalbert}. In the financial compliance scenario, Breden et al. (2020) pointed out that artificial intelligence can be used to preliminarily screen whether the design of financial products complies with regulatory guidance, but generally faces problems such as "language incompatibility" and "lack of explainability".

At the same time, the Retrieval-Augmented Generation (RAG) technology is widely regarded as a solution for knowledge-intensive question answering \cite{lewis2020rag}. This method introduces an external document semantic retrieval module (such as Dense Passage Retrieval, BM25 vector index, etc.) before answer generation, combined with generative models such as BART, T5 or Llama to achieve high-accuracy answers, and has been successfully used in tasks such as Wikipedia question answering \cite{karpukhin2020dense} and medical diagnosis (Nori et al., 2023). However, the introduction of RAG technology into compliance review, especially with the structured regulatory manual published by the UK FCA as the target text, is still a blank.

Specifically speaking of the new regulatory module of "consumer obligations", although it has aroused a lot of discussion in the financial community since it was proposed by the FCA in 2022, academic research is still relatively scattered. Most existing studies focus on macro perspectives such as the institutional costs of compliance transformation and changes in internal audit processes of enterprises (FCA, 2023; Stevenson, 2023), and there is still a lack of structured analysis and semantic modeling around its specific provisions. There is no prototype of a question-and-answer system based on the LLM+RAG framework and with clauses as the retrieval object. The gap in this field provides an entry point for theoretical innovation and technical implementation for this study.

In addition, DeepSeek-V3 is one of the most powerful semantic search architectures among open source models in recent years. It performs well in multi-round long text matching and legal document parsing tasks. Llama 3 released by Meta has also been pre-trained and optimized for instruction understanding, making it more suitable for regulatory style question answering \cite{meta2024llama3}. Combining these two to build a consulting platform for FCA's Consumer Obligations not only has theoretical innovation significance, but also has industry implementation value.

\section{Programme and Methodology}

This study intends to use the Retrieval-Augmented Generation (RAG) architecture, combined with the semantic retrieval model DeepSeek-V3 and the generative language model Llama 3, to build a compliance intelligent consulting platform for automatically understanding, extracting and answering the "Consumer Obligations" clauses of the UK Financial Conduct Authority (FCA). The main reason for choosing this method is that the FCA regulatory text is a highly specialized regulatory language with complex expressions and frequent cross-references. Traditional keyword- or rule-based search and interpretation tools cannot effectively identify semantic reference relationships and contextual dependencies, resulting in inaccurate retrieval and delayed compliance interpretation \cite{lewis2020rag}. The RAG architecture effectively integrates the two capabilities of information location and language understanding by introducing a vectorized semantic retriever and a context-aware generative model. It has been proven to have excellent adaptability and generalization in fields such as medicine (Nori et al., 2023), law \cite{chalkidis2020legalbert}, and knowledge-intensive question answering \cite{karpukhin2020dense}.

In this project, DeepSeek-V3 will undertake the task of semantic retrieval of regulatory clauses. Based on a dual-tower semantic indexing structure, the model can complete the semantic matching between user natural language queries and regulatory paragraphs without relying on explicit labels \cite{deepseek2023}. Unlike traditional models such as BM25, DeepSeek-V3 can understand cross-sentence semantics and contextual synonymous variants, and is more suitable for recall tasks of complex issues in legal scenarios. At the same time, Llama 3, an open large language model released by Meta in 2024, performs well in tasks such as structural prompts, role substitution, and generalization of regulatory language, and can convert the retrieved original clauses into clear and explainable suggestions and answers \cite{meta2024llama3}. After fine-tuning with instructions, the model has high accuracy and style consistency in generation tasks such as multi-round question and answer and compliance report summaries, and is particularly suitable for processing professional corpora.

The research will innovate around two core dimensions: the first is methodological integration innovation, that is, the first integration of DeepSeek-V3 and Llama 3 for structured semantic retrieval and generative compliance question-answering of financial regulatory texts; the second is the expansion of application scenarios innovation, that is, the first systematic application of RAG technology to the FCA's "Consumer Obligations" clause parsing task, filling the current academic and industrial application technology gap in this field \cite{zhang2023regtech}.

During the research process, the entire work is planned to be divided into four interrelated stages: first, the regulatory corpus will be organized and structured, mainly including the extraction, annotation and chapter division of consumer obligation clauses; second, DeepSeek-V3 will be deployed on the corpus to build a regulatory semantic retrieval module, and recall and precision evaluation will be performed; the third stage will develop a generation component based on Llama 3, build task templates and interaction logic, and optimize the stability and interpretability of the model in regulatory context generation; finally, system integration, performance testing and user feedback collection will be carried out to evaluate the platform's response speed, content accuracy and user satisfaction in actual question-and-answer tasks.

The model training process will be strictly based on the FCA's public regulatory text and will not involve private or sensitive data. At the same time, in order to ensure the accuracy and credibility of the model-generated content, a credibility threshold will be set at the answer end to mark and warn of possible "hallucination generation" problems \cite{ji2023hallucination}. The scope of this study is limited to the "Consumer Obligations" related content in the "FCA Handbook" and does not involve other items such as anti-money laundering, capital adequacy ratio or financial technology supervision. Therefore, its extrapolation ability still needs to be further verified in future research.

To assess potential risks, the project has set mitigation strategies for technical risks, computing bottlenecks, and misuse of legal interpretation, such as model quantization compression to cope with video memory limitations, outputting disclaimers to deal with legal misinterpretations, and regularly updating regulatory data to ensure the validity of platform data. In terms of ethics, this study does not collect any personal data or user behavior information. The model training and evaluation are based on open text corpus throughout the process, which meets the requirements of research ethics. Although the platform may generate content with compliance interpretation functions, this system will clearly mark it as an auxiliary analysis tool and does not constitute a formal legal opinion \cite{binns2018responsible}.

In summary, this study not only integrates and innovates the current legal text processing technology in terms of technical methods, but also closely follows the latest trends in the UK's financial regulatory reform in terms of application goals, and has clear research significance and industry value expectations.

\section{Evaluation}

In order to systematically evaluate the effectiveness and practical applicability of the FCA consumer obligations intelligent consultation platform constructed in this study, this study will adopt a hybrid evaluation method, including quantitative indicator analysis and qualitative result interpretation. First, data collection will be mainly based on a self-constructed regulatory corpus and a manually designed question set. The corpus comes from all chapters related to consumer obligations in the "FCA Handbook" publicly released on the FCA official website; the question set refers to the FCA guidelines and real compliance scenarios, simulating about 100 typical question-and-answer tasks, covering key application areas such as information disclosure, customer communication, and product review \cite{fca2023}.

At the quantitative level, the platform's retrieval module (DeepSeek-V3) will be evaluated using criteria such as precision and recall. This type of evaluation will refer to the conventional indicator system in open-domain question-answering systems \cite{karpukhin2020dense}. In terms of the generation module (Llama 3), the consistency of the generated content with the reference answer will be evaluated using a common text generation indicator. The final score will be compared with the output of a traditional search-based regulatory retrieval system to verify the improvement of the RAG architecture platform proposed in this study in terms of accuracy and user understanding \cite{zhang2023regtech}.

In terms of qualitative analysis, typical question and answer samples will be selected for in-depth case analysis, and the performance of the model under different question types will be structured and evaluated in combination with the reference content, context extraction, and prompt style response capabilities of the generated text. Combined with the "illusion risk assessment" method pointed out by Ji et al. (2023), this study will also conduct subjective scoring and manual comparison on the credibility of the model output, clarify the specific triggering conditions that may cause misjudgment of the model, and explore the design path of the avoidance strategy.

Through the above multi-angle hybrid evaluation mechanism, the study will comprehensively present the performance of the platform in terms of accuracy, comprehensibility, usability, stability, etc., so as to verify the technical feasibility and implementation prospects of the compliance consulting platform.

\section{Expected Outcomes}

This study aims to build an intelligent consulting platform based on a retrieval-augmented generation architecture that can automatically handle semantic retrieval and generative question-answering tasks for the UK Financial Conduct Authority (FCA) Consumer Obligations clause. By integrating DeepSeek-V3 and Llama 3, this study is expected to prove that large language models not only have the ability to semantically understand complex regulatory texts in the context of financial compliance, but can also generate effective and explainable compliance recommendations through context awareness, thereby improving the efficiency and accuracy of compliance consulting. The study will verify whether the retrieval-augmented generation (RAG) method is superior to traditional rule matching and keyword search systems, especially in terms of performance differences when dealing with regulatory structured language and question-answering interactive tasks.

On the theoretical level, the expected results of this study will promote academic research in the intersection of artificial intelligence and legal/financial regulation. Existing research focuses on general legal question-answering systems, automatic contract review, or the applicability of models in case prediction (Chalkidis et al., 2020; Zhong et al., 2020), while research on semantic question-answering for structured regulatory provisions, especially for Consumer Obligations, is still blank \cite{zhang2023regtech}. This study not only fills the gap in this application field, but also provides a methodological template for future regulatory analysis tasks.

In practice, the research results will provide financial institutions with a set of deployable and reusable compliance assistance tool frameworks, which are expected to alleviate the compliance resource shortage and understanding gap problems faced by enterprises in the face of regulatory reforms. The platform's open architecture can also be extended to other regulatory topics of the FCA, such as anti-money laundering (AML), financial product suitability review, etc., thereby establishing a blueprint for an intelligent question-and-answer platform for multi-field regulatory compliance.

More importantly, this study is expected to provide empirical data and an actionable paradigm for the current controversial topic of "whether generative AI is suitable for high-precision, high-responsibility professional tasks". By systematically evaluating the performance of DeepSeek-V3 and Llama 3 in real compliance tasks, the study will help clarify the boundary conditions and risk control measures of large language models in highly precise language scenarios such as policy documents and financial supervision. This will not only help promote the application and exploration of AI technology in regulatory technology (RegTech), but will also provide a reference for the applicability of technology for relevant policymakers and regulators.

\section{Research Plan, Milestones and Deliverables}

\definecolor{barblue}{RGB}{153,204,254}
\definecolor{groupblue}{RGB}{51,102,254}
\definecolor{linkred}{RGB}{165,0,33} 

\begin{figure}[htbp]
\begin{ganttchart}[
    y unit title=0.4cm,
    y unit chart=0.5cm,
    vgrid,hgrid,
    x unit=1.55mm,
    time slot format=isodate,
    title/.append style={draw=none, fill=barblue},
    title label font=\sffamily\bfseries\color{white},
    title label node/.append style={below=-1.6ex},
    title left shift=.05,
    title right shift=-.05,
    title height=1,
    bar/.append style={draw=none, fill=groupblue},
    bar height=.6,
    bar label font=\normalsize\color{black!50},
    group right shift=0,
    group top shift=.6,
    group height=.3,
    group peaks height=.2,
    bar incomplete/.append style={fill=green}
   ]{2025-06-01}{2025-08-16}
   \gantttitlecalendar{month=name}\\
   \ganttbar[
    progress=100,
    bar progress label font=\small\color{barblue},
    bar progress label node/.append style={right=4pt},
    bar label font=\normalsize\color{barblue},
    name=pp
   ]{Background Reading}{2025-05-23}{2025-06-07} \\
\ganttset{progress label text={}, link/.style={black, -to}}
\ganttgroup{Work Package 1}{2025-06-08}{2025-06-30} \\
\ganttbar[progress=4, name=T1A]{Task A:Data Preparation}{2025-06-08}{2025-06-15} \\
\ganttlinkedbar[progress=0]{Task B:DeepSeek-V3 Setup}{2025-06-16}{2025-07-01} \\
\ganttgroup{Work Package 2}{2025-07-02}{2025-07-30} \\
\ganttbar[progress=15, name=T2A]{Task A:Llama 3 QA Module}{2025-07-02}{2025-07-20} \\
\ganttlinkedbar[progress=0]{Task B: RAG Integration}{2025-07-21}{2025-07-30} \\
\ganttgroup{Dissertation}{2025-07-30}{2025-08-16} \\
  \ganttbar[progress=0]{Writing and Submission}{2025-07-30}{2025-08-16}
  \ganttset{link/.style={green}}
  \ganttlink[link mid=.4]{pp}{T1A}
  \ganttlink[link mid=.159]{pp}{T2A}
\end{ganttchart}
\caption{Gantt Chart of the activities defined for this project.}
\label{fig:gantt}
\end{figure}

\begin{table}[htbp]
    \begin{center}
        \begin{tabular}{|c|c|l|}
        \hline
        \textbf{Milestone} & \textbf{Week} & \textbf{Description} \\
        \hline
        $M_1$ & 2 & Feasibility study completed \\
        $M_2$ & 5 & Deepseek-V3 Module completed \\
        $M_2$ & 7 & Llama 3 Module completed \\
        $M_3$ & 8 & Intergration and Evaluation completed \\
        $M_4$ & 10 & Submission of dissertation \\
        \hline
        \end{tabular} 
    \end{center}
    \caption{Milestones defined in this project.}
    \label{fig:milestones}
\end{table}

\begin{table}[htbp]
    \begin{center}
        \begin{tabular}{|c|c|l|}
        \hline
        \textbf{Deliverable} & \textbf{Week} & \textbf{Description} \\
        \hline
        $D_1$ & 6 & Software tool for FCA Consumer Duty\\
        $D_2$ & 8 & Evaluation report on performance and usability \\
        $D_3$ & 10 & Dissertation \\
        \hline
        \end{tabular} 
    \end{center}
    \caption{List of deliverables defined in this project.}
    \label{fig:deliverables}
\end{table}

\ Reference
%                Now build the reference list
\bibliographystyle{unsrt}   % The reference style
%                This is plain and unsorted, so in the order
%                they appear in the document.

{\small
\bibliography{main}       % bib file(s).
}

pdflatex main.tex     
bibtex main          
pdflatex main.tex   
pdflatex main.tex    


\end{document}

